
%\documentclass[jgrga]{agutex}
%\documentclass[draft]{agutex}
%\documentclass[12pt]{article}
\documentclass[draft, jgrga]{agutex}

%\documentclass[jgrga]{agutex}
%\usepackage{graphics}
\usepackage{url}
\usepackage[missing={No metadata found}]{gitinfo}

% Two-columned format can be used to estimate the number of pages
% for the final published PDF.


\authorrunninghead{HARRI ET AL.}
\titlerunninghead{MSL HUMIDITY OBS - FIRST RESULTS (version \gitAuthorIsoDate{})}

%Corresponding author mailing address and e-mail address:
\authoraddr{Corresponding author: A.-M. Harri,
Division of Earth Observation. Finnish Meteorological Inst., Finland.
(Ari-Matti.Harri@fmi.fi)}





\begin{document}

%\linenumbers

\title{MSL HUMIDITY OBSERVATIONS - FIRST RESULTS (version \gitAbbrevHash{} on \gitAuthorIsoDate{} by \gitAuthorName)}



\authors{A.-M.~Harri, \altaffilmark{1}
M.~Genzer, \altaffilmark{1}
O.~Kemppinen, \altaffilmark{1}
J.~Gomez-Elvira, \altaffilmark{2}
R.~Haberle,  \altaffilmark{3}
J.~Polkko, \altaffilmark{1}
H.~Savij\"{a}rvi, \altaffilmark{1}
N.~Renn\'{o}. \altaffilmark{9}
J.~A.~Rodriguez-Manfredi, \altaffilmark{2}
W.~Schmidt, \altaffilmark{1}
M.~Richardson,  \altaffilmark{4}
T.~Siili, \altaffilmark{1}
M.~Paton, \altaffilmark{1}
M.~De La Torre-Juarez,  \altaffilmark{5}
T.~M\"{a}kinen, \altaffilmark{1}
C.~Newman, \altaffilmark{4}
S.~Rafkin,  \altaffilmark{6}
M.~Mischna,  \altaffilmark{5}
S.~Merikallio, \altaffilmark{1}
H.~Haukka, \altaffilmark{1}
J.~Martin-Torres, \altaffilmark{2}
M.~Komu, \altaffilmark{1}
M.-P.~Zorzano, \altaffilmark{2}
V.~Peinado, \altaffilmark{2}
L.~Vazquez. \altaffilmark{8}
and R.~Urqui, \altaffilmark{2}}

% A.~Lapinette, \altaffilmark{2}
% and A.~Scodary, \altaffilmark{4}}

\altaffiltext{1}{Finnish Meteorological Institute, Helsinki, Finland.}
\altaffiltext{2}{Centro de Astrobiologia, Madrid, Spain.}
\altaffiltext{3}{NASA AMES research center, San Francisco, US.}
\altaffiltext{4}{Ashima Research Inc., Pasadena, US.}
\altaffiltext{5}{NASA Jet Propulsion Laboratory, Pasadena, US.}
\altaffiltext{6}{Southwest Research Institute, Boulder, US.}
\altaffiltext{7}{Texas A \& M University, Tx, US.}
\altaffiltext{8}{University of Complutense, Madrid, Spain}
\altaffiltext{9}{University of Michigan, US.}

%% The [] brackets identify the author to the corresponding affiliation, 1, 2, 3, etc. should be inserted.


%\received{}
%\pubdiscuss{} %% only important for two-stage journals
%\revised{}
%\accepted{}
%\published{}

%% These dates will be inserted by the Publication Production Office during the typesetting process.


\begin{abstract}
The Mars Science laboratory (MSL) called Curiosity made a successful landing at Gale crater early August 2012. MSL has an environmental instrument package called the Rover Environmental Monitoring Station (REMS) as a part of its scientific payload. REMS comprises instrumentation for the observation of atmospheric pressure, temperature of the air, ground temperature, wind speed and direction, relative humidity (REMS-H), and UV measurements. We concentrate on describing the measurement performance and first results of the REMS relative humidity observations during the first 100 MSL sols and constrain the REMS-H results by comparing them with indirect observations. The REMS-H device is based on polymeric capacitive humidity sensors developed by Vaisala Inc. and it makes use of transducer electronics section placed in the vicinity of the three (3) humidity sensor heads. The humidity device is mounted on the REMS boom providing ventilation with the ambient atmosphere through a filter protecting the device from airborne dust. The final relative humidity results appear to be convincing and are aligned with earlier indirect observations of the total atmospheric precipitable water content.
\end{abstract}


\begin{article}

\section{Introduction}\label{sec:Intro}

\section{Background}\label{sec:Background}

\subsection{Section title TBD}

\subsection{Observations}

\subsection{Modeling and the characteristics of the Martian water cycle}



\section{REMS-H Calibration}\label{sec:REMSHCalibration}


\section{REMS-H operations and performance}\label{sec:REMSHOpsPerf}


\section{REMS-H Observations and First results}\label{sec:REMSHObsResults}



\section{Concluding remarks and discussion}\label{sec:Conclusions}



\bibliographystyle{agufull08}
\bibliography{REMSHrefs}

\begin{acknowledgments}

The authors would like to express their gratitude to the MSL and REMS instrument teams in making this wonderful Mars mission come true. Ari-Matti Harri and Hannu Savijarvi are thankful for the Finnish Academy grants No 1241212312 and  No 3452352345.

\end{acknowledgments}


\end{article}
\end{document}
